\documentclass[12pt,a3paper]{article}
\usepackage[utf8]{inputenc}
\usepackage[german]{babel}
%\usepackage[T1]{fontenc}
\usepackage{amsmath}
\usepackage{amsfonts}
\usepackage{amssymb}
\usepackage{amsthm}
\usepackage{makeidx}
\usepackage{graphicx}
\usepackage[left=2.5cm,right=2.5cm,top=2cm,bottom=2cm]{geometry}
\author{Elena Kleinwort}

\newtheorem{theorem}{Aussage}
\newtheorem{teil}{Teil}
\begin{document}

\renewcommand{\arraystretch}{1.2}

\begin{theorem}

Der durch 4 Punkte auf der Einheitssphäre $\mathbb{S}^3$ definierte sphärische Tetraeder hat ein größeres Volumen als der durch die gleichen Punkte definierte euklidische Tetraeder. (Und das müsste sich eigentlich auf alle dimensionen verallgemeinern lassen.)

\end{theorem}

\begin{teil}
Der Abstand von jedem Punkt des euklidischen Tetraeders zum Nullpunkt ist $ \leq 1$.
\end{teil}

\begin{proof}
Jeder Punkt auf der Kante zwischen den Punkten $v_1$ und $v_2$ lässt sich als $v_1 + (v_2 - v_1)\cdot t = (1-t) v_1 + t v_2 $. Die Norm dieses Vektors ist:

\begin{eqnarray}
\begin{array}{rll}
\lVert (1-t) v_1 + t v_2 \rVert  =& ((1-t) x_1 + t  x_2)^2 + ((1-t) y_1 + t  y_2)^2 + ((1-t) z_1 + t  z_2)^2 &\\
 =& (1-t)^2x_1^2 + t^2x_2^2 + (1-t)tx_1x_2   
 +  (1-t)^2y_1^2 + t^2y_2^2 + (1-t)ty_1y_2  
 +  (1-t)^2z_1^2 + t^2z_2^2 + (1-t)tz_1z_2  &\\
 = & (1-t)^2 \lVert v_1 \rVert + t^2 \lVert v_2 \rVert  + (1-t)t \langle v_1, v_2 \rangle &\\
 = & (1-t)^2 + t^2 + (1-t)t \langle v_1, v_2 \rangle &\\
 = & 1 - 2t + 2t^2 + (t - t^2) \langle v_1, v_2 \rangle &\\
 = & 1 - 2(t - t^2) + (t - t^2) \langle v_1, v_2 \rangle &\\
 = & 1 + (t - t^2) (\langle v_1, v_2 \rangle -2) &\\
 \leq & 1 + (t - t^2) (1 -2) &\\
 = & 1 - (t - t^2) \leq 1&\\
\end{array}
\end{eqnarray}

Alle anderen Punkte liegen auf Strecken zwischen Punkten auf diesem Kanten. Mit dem gleichen Argument Ergibt sich die Behaupttung somit für alle Punkte des euklidischen Tetraeders.

\end{proof}

 


\end{document}